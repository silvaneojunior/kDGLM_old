% Options for packages loaded elsewhere
\PassOptionsToPackage{unicode}{hyperref}
\PassOptionsToPackage{hyphens}{url}
%
\documentclass[
]{article}
\usepackage{amsmath,amssymb}
\usepackage{lmodern}
\usepackage{iftex}
\ifPDFTeX
  \usepackage[T1]{fontenc}
  \usepackage[utf8]{inputenc}
  \usepackage{textcomp} % provide euro and other symbols
\else % if luatex or xetex
  \usepackage{unicode-math}
  \defaultfontfeatures{Scale=MatchLowercase}
  \defaultfontfeatures[\rmfamily]{Ligatures=TeX,Scale=1}
\fi
% Use upquote if available, for straight quotes in verbatim environments
\IfFileExists{upquote.sty}{\usepackage{upquote}}{}
\IfFileExists{microtype.sty}{% use microtype if available
  \usepackage[]{microtype}
  \UseMicrotypeSet[protrusion]{basicmath} % disable protrusion for tt fonts
}{}
\makeatletter
\@ifundefined{KOMAClassName}{% if non-KOMA class
  \IfFileExists{parskip.sty}{%
    \usepackage{parskip}
  }{% else
    \setlength{\parindent}{0pt}
    \setlength{\parskip}{6pt plus 2pt minus 1pt}}
}{% if KOMA class
  \KOMAoptions{parskip=half}}
\makeatother
\usepackage{xcolor}
\IfFileExists{xurl.sty}{\usepackage{xurl}}{} % add URL line breaks if available
\IfFileExists{bookmark.sty}{\usepackage{bookmark}}{\usepackage{hyperref}}
\hypersetup{
  pdftitle={Relatorio de mudanças no código},
  pdfauthor={Silvaneo Viera dos Santos Junior},
  hidelinks,
  pdfcreator={LaTeX via pandoc}}
\urlstyle{same} % disable monospaced font for URLs
\usepackage[margin=1in]{geometry}
\usepackage{graphicx}
\makeatletter
\def\maxwidth{\ifdim\Gin@nat@width>\linewidth\linewidth\else\Gin@nat@width\fi}
\def\maxheight{\ifdim\Gin@nat@height>\textheight\textheight\else\Gin@nat@height\fi}
\makeatother
% Scale images if necessary, so that they will not overflow the page
% margins by default, and it is still possible to overwrite the defaults
% using explicit options in \includegraphics[width, height, ...]{}
\setkeys{Gin}{width=\maxwidth,height=\maxheight,keepaspectratio}
% Set default figure placement to htbp
\makeatletter
\def\fps@figure{htbp}
\makeatother
\setlength{\emergencystretch}{3em} % prevent overfull lines
\providecommand{\tightlist}{%
  \setlength{\itemsep}{0pt}\setlength{\parskip}{0pt}}
\setcounter{secnumdepth}{-\maxdimen} % remove section numbering
\ifLuaTeX
  \usepackage{selnolig}  % disable illegal ligatures
\fi

\title{Relatorio de mudanças no código}
\author{Silvaneo Viera dos Santos Junior}
\date{5/19/2022}

\begin{document}
\maketitle

\hypertarget{mudanuxe7as-gerais-exceto-caso-normal}{%
\subsection{Mudanças gerais (exceto caso
Normal)}\label{mudanuxe7as-gerais-exceto-caso-normal}}

\begin{itemize}
  \item Limpeza do código: remoção de variáveis e comentários não utilizados.
  \item Simplificação do loop de filtragem: todo o processo definido em um loop (originalmente a primeira iteração era executada fora do loop).
  \item Otimização do loop de filtragem: minimização de produtos matriciais e de chamadas de funções \emph{gamma}, \emph{digamma} e \emph{trigamma}.
  \item Separação das funções de filtragem, predição, suavização e ajuste.
  \item Otimização do loop de suavização: minimização da quantidade de inversões de matrizes.
  \item Inclusão da possibilidade de variáveis com variância nula: durante a suavização, estas variáveis são removidas das operações.
  \item Mudança na matriz $R_t$, inicialmente definida como $\frac{1}{\delta}P_t$, onde $P_t=G C_t G'$ e $\delta$ é o fator de desconto (não precisa ser escalar no código, mas por simplicidade, escreverei como um escalar). Na versão atual, temos $R_t=\frac{1}{\delta}P_t+W_t$, onde $W_t$ é uma simétrica definida não-negativa (falta incluir um erro para quando esta restrição não é respeitada pelo usuário). A ideia dessa mudança veio da observação do fato de que, na especificação original, se $C_0=0$ (variância à priori), então $C_t=0$ para todo $t$, com a inclusão da matriz $W_t$, é possível que uma variável tenha variância $0$ até um certo tempo e depois receba uma choque aleatória.
  \item Padronizações entre os códigos para o caso Poisson (com Linear Bayes) e Multinomial.
\end{itemize}

\hypertarget{caso-multinomial}{%
\subsection{Caso multinomial}\label{caso-multinomial}}

\begin{itemize}
  \item Generialização do código para $k$ categorias (conforme o artigo no arxiv).
  \item Extensão da $F_t$ para regressão dinâmica. No original, $F_t=F$ para todo $t$ (série temporal).
  \item $F_t$ se tornou um array de dimensão $n \times r \times T$ ($n$ é o número de variáveis latentes, $r$ é a quantidade de series na saída do modelo e $T$ é o tempo final). Para cada tempo, temos uma matriz $n \times r$, sendo que o preditor linear está definido como $F_t' \theta_t$.
  \item $F_t$ não é mais bloco diagonal, é possível definir variáveis latentes que tem efeito em mais de uma série. Desvantagem: um pouco mais complicado para o usuário fazer a especificação, porém algumas funções auxiliares foram criadas para facilitar o processo, além disso, a aplicação do \emph{Shiny} simplifica o processo.
  \item Generização da condição inicial para o Newton-Raphson (garantindo inicialização válida): $\vec{x}_{inicial}=(0.01,...,0.01,0.01 \times r)$. Essa inicialização garante que $x_{r+1}-\sum_{i=1}^{r} x_i>0$ ($x_{r+1}$ está associado ao total no sistema).
\end{itemize}

\hypertarget{caso-normal}{%
\subsection{Caso Normal}\label{caso-normal}}

Esse script recebeu o mínimo de alteranções necessárias para que o
ajuste funcionasse, de modo a evitar introduzir erros antes que resolver
os que já estavam presentes

\begin{itemize}
  \item Correções gerais para igualar o código e as equações do artigo (as mudanças já tinham sido quase integralmente observadas pela Mariane anteriormente).
  \item Mudança de variáveis do sistema de compatibilização da priori: o sistema passa a ser escrito em termos de $\mu_0=-\frac{\tau_2}{\tau_1}$, $c_0=-2\tau_1$, $d_0=\frac{\tau_2^2}{2\tau_1}-2\tau_3$ e $n_0=2\tau_0+1$.
  \item Simplificação do sistema (mais detalhes no final).
\end{itemize}

Sistema original:

\[
\begin{align}
\frac{(2\tau_0+1)\tau_2^2}{2\tau_1\tau_2^2-8\tau_1^2\tau_3}-\frac{1}{2\tau_1} & = (q_1+f1^2)\exp(f_2+q_2/2)\\
-\frac{(2\tau_0+1)\tau_2}{\tau_2^2-4\tau_1\tau_3} & = f1\exp(f_2+q_2/2)\\
\frac{4\tau_1(\tau_0+1/2)}{\tau_2^2-4\tau_1\tau_3} & = \exp(f_2+q_2/2)\\
\gamma(\tau_0+1/2)-\log(\frac{\tau_2^2}{4\tau_1-\tau_3}) & = f_2
\end{align}
\]

Observe que:

\[
-\frac{(2\tau_0+1)\tau_2}{\tau_2^2-4\tau_1\tau_3}=-\frac{2\tau_2}{4\tau_1}\frac{4\tau_1(\tau_0+1/2)}{\tau_2^2-4\tau_1\tau_3}
\]

Usando a equação \(3\), temos:

\[
-\frac{(2\tau_0+1)\tau_2}{\tau_2^2-4\tau_1\tau_3}=-\frac{2\tau_2}{4\tau_1}\frac{4\tau_1(\tau_0+1/2)}{\tau_2^2-4\tau_1\tau_3}=-\frac{2\tau_2}{4\tau_1}\exp(f_2+q_2/2)=f1 \exp(f_2+q_2/2)
\]

Daí: \[
-\frac{2\tau_2}{4\tau_1}=\mu_0=f1 
\]

Veja agora que:

\[
\frac{(2\tau_0+1)\tau_2^2}{2\tau_1\tau_2^2-8\tau_1^2\tau_3}=\frac{-\tau_2}{2\tau_1}\frac{(2\tau_0+1)\tau_2}{\tau_2^2-4\tau_1\tau_3}
\]

Usando a equação \(2\) e que \(-\frac{\tau_2}{2\tau_1}=\mu_0=f_1\),
temos:

\[
\frac{(2\tau_0+1)\tau_2^2}{2\tau_1\tau_2^2-8\tau_1^2\tau_3}=\frac{-\tau_2}{2\tau_1}\frac{(2\tau_0+1)\tau_2}{\tau_2^2-4\tau_1\tau_3}=f_1^2\exp(f_2+q_2/2)
\]

Voltando à equação \(1\):

\[
\frac{(2\tau_0+1)\tau_2^2}{2\tau_1\tau_2^2-8\tau_1^2\tau_3}-\frac{1}{2\tau_1} = f_1^2\exp(f_2+q_2/2) -\frac{1}{2\tau_1} = (q_1+f1^2)\exp(f_2+q_2/2)
\]

Logo:

\[
 -\frac{1}{2\tau_1} = (q_1+f1^2)\exp(f_2+q_2/2) - f_1^2\exp(f_2+q_2/2)=q_1\exp(f_2+q_2/2)
\]

\[
 \frac{1}{c_0} = q_1\exp(f_2+q_2/2)
\]

\[
c_0 = \frac{1}{q_1\exp(f_2+q_2/2)}
\]

Assim, as duas primeiras equações do sistema possuem solução analítica e
que não depende de \(n_0\) e \(d_0\). Vamos agora para a equação \(3\).
Usando que \(\frac{n_0}{2}=\tau_0+1/2\) e que
\(\frac{4\tau_1}{\tau_2^2-4\tau_1\tau3}=\frac{1}{\frac{\tau_2^2}{4\tau_1}-\tau_3}=\frac{1}{\frac{d_0}{2}}\):

\[
\frac{4\tau_1(\tau_0+1/2)}{\tau_2^2-4\tau_1\tau_3} = \frac{n_0/2}{d_0/2}= \frac{n_0}{d_0} = \exp(f_2+q_2/2)
\]

Ademais, na equação \(4\), temos:

\[
\gamma(\tau_0+1/2)-\log\left(\frac{\tau_2^2}{4\tau_1-\tau_3}\right) = \gamma\left(\frac{n_0}{2}\right)-\log\left(\frac{d_0}{2}\right)= f_2
\]

Com a equação \(3\), podemos escrever \(d_0\) como função linear de
\(n_0\), fazendo essa substituição na equação \(4\), obtemos:

\[
 \gamma\left(\frac{n_0}{2}\right)-\log\left(\frac{n_0}{2\exp(f_2+q_2/2)}\right)= f_2
\]

Resolvemos o sistema acima usando Newton-Raphson e usando
\(x=\log(n_0)\) como argumento do sistema para garantir \(n_0>0\):

\[
 \gamma\left(\frac{e^x}{2}\right)-x+\log(2)+f_2+q_2/2= f_2
\]

Ou melhor:

\[
 \gamma\left(\frac{e^x}{2}\right)-x+\log(2)+q_2/2=0
\]

É importante ressaltar que a solução deste sistema não depende de
\(f_2\), ademais, o processo de resolução é bem estável numericamente,
sendo resolvivel mesmo para valores extremos de \(q_2\).

Uma vez obtida o valor de \(n_0\), temos o valor de \(d_0\) através da
relação \(d_0=\frac{n_0}{\exp(f_2+q_2/2)}\). Problema: se \(f_2+q_2/2\)
for muito grande, \(\exp(f_2+q_2/2)\) pode ser computacionalmente
intratável, porém, para qualquer valor de \(q_2\), é possível fazer uma
mudança de escala nos dados de maneira que \(\exp(f_2+q_2/2)\) se torne
tratável, pois ao multiplicarmos os dados por uma constante \(c\),
obtemos:

\[
f^*_2=f^*_2-\log(c),
\] pois \(f_2\) é a \(\log\) precisão, ademais, \(q_2\) não muda de
valor, pois a variância não é afetada pela soma/subtração de constantes.

\end{document}
